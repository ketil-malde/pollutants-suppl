\documentclass[11pt,a4paper]{article}

\usepackage[utf8]{inputenc}
\usepackage[T1]{fontenc}
\usepackage[top=0.85in,left=1in,footskip=0.75in]{geometry}
\usepackage{graphicx}
\usepackage{amsmath}

% xetex-specific(?), needed for Greek literals in text (i.e., Sigma)
\usepackage{fontspec}
\usepackage{xltxtra,xunicode}
\defaultfontfeatures{Mapping=tex-text,Scale=MatchLowercase}
\setromanfont{TeX Gyre Pagella}

\title{Supplementary data}
\author{}
\date{}

\begin{document}
\maketitle

Statistical analyses of the different contaminants in North East
Atlantic cod and haddock liver samples.  All analyses were performed
using the \texttt{python} programming language.  Figures were created
using \texttt{matplotlib}, and regression was performed using the
\texttt{OLS} method from the \texttt{statsmodels} package.  The
\texttt{pandas} library was used for data import and manipulation.

Contamination level $C$ in each case as was modeled as a function of
fish weight $w$ and year $y$.

\begin{equation*}
ln(C) \sim \alpha ln(w) + \beta y + k
\end{equation*}

In some cases samples from specific years appear to be correlated,
probably due to differences in the sampled populations.  This implies
heteroscedacity in the residuals. While this does not cause bias in
the regression coefficients, it can result in values for $p$ and $R^2$
that are smaller than they should be, and also overly optimistic
confidence interval estimates.  Some caution is therefore adviced in
interpreting these results.

% source: http://statisticsbyjim.com/regression/heteroscedasticity-regression/

